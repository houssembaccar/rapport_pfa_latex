\chapter*{Conclusion et Perspectives}

Notre projet se situe dans le cadre de l'optimisation du processus de recherche des services Cloud à travers la manipulation des ontologies du Cloud et le calcul de similarité.
C'est dans cet esprit là que nous allons dresser un bilan de notre travail et présenter quelques perspectives future.\\

\textbf{Bilan et contributions}

En vue de réaliser l'objectif qu'on s'est imposé , nous avons divisé la charge de travail en plusieurs étapes.
Nous avons donc procédé à la collecte de documents , articles , théses traitant le sujet ainsi que les projets existants.
Nous nous sommes donc principalement interessé au fonctionnement du moteur de recherche de services Cloud : "Cloudle " du fait qu'il se base sur la consultation et la manipulation d'une antologie du web.
Cependant même si "Cloudle" traite la similarité de concepts , il considére les propriétés existantes tel que le stockage mémoire ,la vitesse du processeur comme étant des concepts apparentières ce qui rend la recherche moins performante qu'elle aurait dû être.
Alors que la recherche avec "Cloudle" repose sur le calcul de similarité entre concepts et sur le raisonnement équivalent et numérique entre les concepts , nous avons décidé de partir sur une approche partiellement différente de celle utilisée par "Cloudle".
En effet , en plus d'avoir recours à la consultation d'une ontologie du Cloud et du calcul de similarité de concepts , notre approche consiste dans le développement d'un algorithme de recherche qui met au point l'utilisation des similarités des propriétés objet et type de données pour ajouter à la précision et l'efficacité de la recherche pour ainsi obtenir non seulement un procédé de recherche dont les résultats sont précis mais répondent à l'attente des utilisateurs.

Ce projet nous a bénéficier tant sur le plan théorique que pratique.En effet , il nous a permit de nous ouvrir sur de tout nouveaux domaines dont nous ignorions tout qui sont le Cloud Computing et Le web sémantique.
De plus , nous avons pû renforcer notre maîtrise du language de programmation Java avec lequel nous avons developpé notre application.\\

\textbf{Perspectives de recherche}
L'algorithme que nous avons developpé , a été testé à plusieurs reprises sur de diverses ontologies et a donné des résultats très satistfaisant .
Cependant, nous n'avons pu étudier que les cas d'ontologies à la structure assez simple , ceci dû à la fois au manque de ressources et de temps à notre disposition.
Cela n'empêche que nous désirons essayer notre algorithme sur des ontologies aux structures beaucoup plus complexes pour comparer les résultats obtenus lors du passage à l'échelle et potentiellement adapter notre travail à l'utilisation dans le marché actuel. 