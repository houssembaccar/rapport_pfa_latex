\documentclass[10pt,a4paper ]{report}
\usepackage[utf8]{inputenc}
\usepackage[francais]{babel}
\usepackage[T1]{fontenc}

\begin{document}
\textbf{Résumé}\\
Avec la croissance de la volumétrie des données, il est devenu opportun de réduire le temps d’accès à ces dernières et d’optimiser la performance des requêtes. Dans ce contexte s’inscrit notre sujet de stage, l’objectif principal étant de chercher les meilleurs attributs indexables et d’en créer la structure d’optimisation adoptée au cours de notre recherche qui consiste en l’index de jointure bitmap IJB.\\
Dans un premier volet, nos critères de choix de ces attributs se basent sur la fréquence d’utilisation et la cardinalité des tables. Pour se faire, on s’est inspiré de travaux déjà existants tels que l’algorithme Close et l’algorithme Dynaclose. Dans un deuxième volet, nous allons constater que les critères cités précédemment traitent les requêtes de façon isolée les unes des autres. Alors, nous avons pensé à ajouter un nouveau critère pour sélectionner les meilleurs attributs qui consiste en la prise en compte de l’interaction entre les requêtes en se basant sur le (Multiple View Processing Plan) MVPP, qui se présente en un plan unifié représentant une visualisation des requêtes et des interactions entre elles.\\
\textbf{Mots Clefs : }Index de Jointure Binaire (IJB), Interaction, Performance\\ \ \\
\textbf{Abstract}\\
With the growth of data’s volume, it has become important to reduce the access time to them and to optimize query performance. In this context, we can say that the main objective of our internship is to seek  the best indexable attributes and create the optimization structure adopted in our research which consists on bitmap join index BJI.\\
In the first part, our selection criteria of these attributes are based on the frequency of use and the tables cardinality. To do so, we was inspired by existing work such as Close algorithm and DynaClose algorithm. In a second part, we note that the criteria mentioned above handle queries in isolation from each other. So we thought of adding a new criterion to select the best attributes which consists in taking into account the interaction between queries. We were based on the (Multiple View Processing Plan) MVPP which consists in a unified tree representing a visualization of queries and interactions between them.\\
\textbf{Key words :} Bitmap Join Index BJI, Interaction, Performance.\\ \ \\
words : Bitmap Join Index BJI, Interaction, Performance.



    
\end{document}