\chapter*{Introduction Générale}
\markboth{Introduction Générale}{}

    \hspace{1.5cm} \textbf{ Contexte et problématique du travail}

        Le "Cloud Computing" ou "L'Informatique en Nuage" peut-être définie comme un paradigme informatique distribué à grande échelle dans lequel on trouve une collection de plateformes, puissance de calcul , espaces de stockages et autres services qui sont dynamiquement évolutifs,virtualisés et qui sont délivrés aux clients à travers Internet.\\
        Cependant, il n'y a pas de standards et de protocoles pour gérer la recherche des différents Clouds d'où la nécessité d'avoir un engin de recherche pour les services Cloud.\\
        Pour augmenter nos chances de découvrir les services clouds appropriés, nous avons besoin de représenter les relations sémantiques entre les différents les concepts du Cloud service en ayant recours à une ontologie du Cloud.\\
        Cette Ontologie nous permettra de  raisonner sur la similarité entre les concepts du Cloud et aboutir à un mécanisme efficace pour la recherche des services Clouds.\\

    \hspace{1.5cm} \textbf{ Contributions  }\\
        Notre projet consiste en la réalisation d'une application java , qui permet de raisonner sur le language OWL (Ontologie Web Language) à travers un algorithme de recherche basé sur le calcul de la similarité entre les différents individus d'une ontologie et ainsi de faciliter la recherche des services Cloud sur le Web en permettant aux utilisateurs de trouver un service Cloud spécifique ou de trouver un service qui est similaire à la requête de l'utilisateur en se basant sur le résultat de la similarité.\\




    \hspace{1.5cm} \textbf{ Organisation du projet }\\
    L’organisation de notre rapport se présente comme suit :\\Dans le premier chapitre, nous allons aborder les définitions des concepts ayant une importance relative à la réalisation du projet  tel que : Le Cloud computing son architecture et ses différentes couches, l’ontologie et les environnements de developpement qui la permettent sa construction, l'OWL, La similarité etc.\\
    Dans le deuxième chapitre  nous allons nous intéresser de plus près à la similarité sur ses trois formes (sur les concepts, sur les propriétés d'objets et sur les propriétés de données) : Leurs formules de calcul et leur complexité algorithmique.\\
    Dans le troisième chapitre  nous allons aborder les étapes d'implémentation de notre projet (Conception, réalisation, test et validations...) et les outils techniques utilisés à cette fin.\\
    Nous concluons ce rapport par un bilan général sur nos contributions et nos constations, et la discussion  des perspectives qui en dérivent.\
