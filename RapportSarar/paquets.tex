%------------------------------------------------------------
% CHARGEMENT DES PACKAGES NECESSAIRES ET DEFINITIONS
%------------------------------------------------------------
\usepackage[T1]{fontenc}
%\usepackage{times}
\usepackage{graphicx}
\usepackage[french]{babel}
\usepackage{overpic}
\usepackage{amssymb}
\usepackage{amsmath,upref}
\usepackage{multirow}
\usepackage{epsfig}
\usepackage{verbatim}
\usepackage{graphicx}
\usepackage{amsmath}
\usepackage{here}
\usepackage{lscape}
\usepackage{fancybox}
\usepackage{fancyhdr}
\usepackage{subeqnarray}
\usepackage{epsfig}
\usepackage{algorithm}
\usepackage{algorithmic}
\usepackage{listings}
%\usepackage[latin1]{inputenc}
%\usepackage{subfigure,relsize}
%\usepackage{algorithm}
%----------------------- Definition des marges ---------------------
\setlength{\parindent}{0pt} \setlength{\hoffset}{-1.5cm}
\setlength{\oddsidemargin}{1.1cm} % Marge gauche sur pages impaires 1.2
\setlength{\evensidemargin}{1.1cm} % Marge gauche sur pages paires 0.6
\setlength{\marginparwidth}{54pt} % Largeur de note dans la marge
\setlength{\textwidth}{16.8cm} % Largeur de la zone de texte (17cm)
\setlength{\voffset}{-1.5cm} % Bon pour DOS
\setlength{\marginparsep}{7pt} % Separation de la marge
\setlength{\topmargin}{0pt} % Pas de marge en haut
\setlength{\headheight}{13pt} % Haut de page
\setlength{\headsep}{1cm} % Entre le haut de page et le texte
\setlength{\footskip}{1.5cm} % Pied de page + separation
\setlength{\topskip}{0cm} % Pied de page + separation
\setlength{\textheight}{23.4cm} % Hauteur de la zone de texte
\setlength{\headwidth}{\textwidth}
%-- ent^ete et pied de page [ version RECTO-VERSO ]----------
\pagestyle{fancy}
%% en-t^ete de page % O : impaire E : paire
\fancyhead{} \fancyhead[LO]{\footnotesize \sffamily \nouppercase{\leftmark}}
\fancyhead[CO]{} \fancyhead[RO,LE]{ } \fancyhead[RE]{\footnotesize \sffamily
\nouppercase{\rightmark}} \fancyhead[CE]{}
%% pied de page
\fancyfoot[LO]{\footnotesize \it Recherche de sous-structures
arborescentes frequentes avec des contraintes d'inclusion floue}
\fancyfoot[RE]{\small \it Ahmed CHEMORI} \fancyfoot[RO,LE]{\thepage}
\fancyfoot[CO]{} \fancyfoot[CE]{}
\renewcommand{\headrulewidth}{0.9pt}
%-------------------------------------------------------------------
\newtheorem{rem}{\bf Remarque}
\newtheorem{hyp}{\bf Hypothese}
\newtheorem{definition}{\bf Definition}
\newtheorem{proposition}{\bf Proposition}
\newtheorem{lem}{\bf Lemme}
%------------------- Definition de nouvelles commandes ----------------
\def\rref#1{{\rm (\ref{#1})}}
\newcommand\mR{\mathbb{R}}
\newcommand{\defeq}{:=}
\newcommand{\mat}[1]{\begin{pmatrix}
#1
\end{pmatrix}}
%--------------------------------
